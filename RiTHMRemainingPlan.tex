\documentclass[]{article}
\usepackage[nodayofweek,level]{datetime}
\usepackage{hyperref}
\newcommand{\rithm}{\textbf{RiTHM}\space}
% Title Page
\title{ Plan for deliverables of \textbf{RiTHM} Development and Journal Paper}
\author{Yogi Joshi}


\begin{document}
\maketitle

\section{Current Results}
Following items have been implemented
\begin{itemize}
	\item \rithm Parser APIs - The interfaces and abstract classes for parser API, which can be called by other \rithm components along with examples of the parser developed using the APIs  including LTL, PTLTL parser which are added to the framework. The code is at \url{https://github.com/yogirjoshi/Parser}
	\item \rithm Monitor APIs - The interfaces and abstract classes for Monitor APIs. Example implementations of the interface include LTL Monitor, MTL monitor (work in progress). The code is at \url{https://github.com/yogirjoshi/MonitorFactory}	
	\item \rithm Data Tools - The APIs allow to input trace data to \rithm in various formats such as XML and CSV. The code is at \url{https://github.com/yogirjoshi/DataFactory}
	\item \rithm Network Interface - Allows to connect to a \rithm instance using a \rithm protocol which is coded over top of TCP. The scripts are being tested. Th\url{https://github.com/yogirjoshi/RiTHMServer}
	\item The above components interact with each other for providing a monitoring framework, and dynamically classes are loaded using Java's class-loader to allow loading plugins (which implement required \rithm interfaces) from the jars. 
	\item \rithm Plots - The offline plots are developed using gglplot2. The scripts to be exposed via web interface to render PNG file in html. Upgrade of this functionality is in progress to use \textit{Shiny} to render plots.
	\item
	\textbf{Thesis} topics associated with the journal paper contents will be written at same time along with journal paper sections 
\end{itemize}
\section{Proposed Schedule}
\subsection{Development Sprint 1}
\textbf{Target Date:}\space \formatdate{7}{5} {2015}\newline
\textbf{Description:} MTL Monitor \& Monitor Plugin using \rithm APIs for MTL (Metric Temporal Logic)\newline
\textbf{Goals:}
\begin{itemize}
\item Develop and integrate Monitor Pluign using \rithm Monitor API
\item Develop Unit Test Cases for above Monitor Plugin
\item Run Test cases with MTL properties for QNX Traces\\
\end{itemize}

\subsection{Development Sprint 2}
\textbf{Target Date:}\space \formatdate{18}{5} {2015}\newline
\textbf{Description:} \rithm Plugin Loader extension,  its GUI and Journal Paper - lessons learnt\newline
\textbf{Goals:}
\begin{itemize}
\item
Develop and integrate Plugin Loader into current framework
\item
Currently, we point JARs and RiTHM Searches for plugin classes in those JARs
\item
Goal is to create a separate plugin-loader factory class which allow selecting plugins and this functionality can be
easily exposed in GUI
\item
Lessons learnt from CSRV'14 competition shall be a section in Journal Paper, Goal is to create a draft of the lessons learnt section; 

\end{itemize}

\subsection{Development Sprint 3}
\textbf{Target Date:}\space \formatdate{29}{5} {2015}\newline
\textbf{Description:} Document \rithm examples for plugin development, and Journal Paper - RiTHM design section\newline
\textbf{Goals:}
\begin{itemize}
	\item
	Create 'Developer's Manual' which contains examples of how to develop different plugins - Monitor, DataTools, Parser, Network Interface.
	\item
	To include examples, documentation of existing framework's APIs, interfaces, abstract classes, class interaction, etc.
	\item
	Create a draft of \rithm's architecture and design for the journal paper. Detailed design of abstract components for addressing Runtime Verification problem and modeling their interaction.
	\item
	Discuss current monitoring tools and methods used in Software Engineering processes and their limitations, and how \rithm addresses these limitations - advantages of \rithm framework over over Test Driven Development methodology, Monitoring Tools for Enterprise Applications; 
	\item
	Develop GUI web components for previous sprints. 
\end{itemize}

\subsection{Development Sprint 4}
\textbf{Target Date:}\space \formatdate{10}{6} {2015}\newline
\textbf{Description:} Add runtime plots to \rithm using matplotlib of python. Currently, we have offline plots which use ggplot2 \newline
\textbf{Goals:}
\begin{itemize}
	\item
	Create python class framework which receives data feed from monitor(possibly over network) and plots the truth-values of formulae over time.
	\item
	Currently, we have RiTHM-Protocol (on top of TCP) for communicating with RiTHM server's components over network. The protocol will be extended to add messages for additional plotting.
	\item
	\rithm journal paper's section to be delivered - Implementation details, introduction, abstract, conclusion
\end{itemize}


\subsection{Development Sprint 5}
\textbf{Target Date:}\space \formatdate{21}{6}{2015}\newline
\textbf{Description:} Add GPU monitoring functionality using JCuda, CUDA wrapper for Java \newline
\textbf{Goals:}
\begin{itemize}
	\item
	Ability to offload predicate evaluation to GPUs
	\item
	Update jorunal version as per feedback
	\item
	Complete development of any of the remaining items in previous sprints.
	\item
	Develop GUI web components for previous sprints. 
\end{itemize}

\subsection{Development Sprint 6}
\textbf{Target Date:}\space \formatdate{31}{6} {2015}\newline
\textbf{Description:} Case-studies' section in \rithm journal paper \newline
\textbf{Goals:}
\begin{itemize}
	\item
	Deliver case-studies for \rithm framework. Monitoring different logics ( LTL, MTL, FOMTL) - online and offline.
	\item
	Possible examples: QNX System Call Traces, Traces from CSRV'14 competitions from various tools, etc.
	\item
	Add case studies to journal paper as well as to the Developers' manual of \rithm
	\item
	Develop GUI web components for previous sprints. 
\end{itemize}

\subsection{Development Sprint 7}
\textbf{Target Date:}\space \formatdate{07}{7} {2015}\newline
\textbf{Description:} \rithm Journal paper submission \newline
\textbf{Goals:}
\begin{itemize}
	\item
	Complete journal paper
	\item
	\rithm 2.0 public release 
	\item
	Release will include Developer's manual, User's manual and screen-cast (with audio) on \rithm's website.

\end{itemize}

\end{document}          
